\problemset{Теория вероятностей и математическая статистика}
\problemset{Индивидуальное домашнее задание №0}	% поменяйте номер ИДЗ

\renewcommand*{\proofname}{Решение}

%%%%%%%%%%%%%% ЗАДАНИЕ №1 %%%%%%%%%%%%%%
%% Условие задания №1
\begin{problem}
	Из урны, в которой лежат $ K $ белых и $ L $ чёрных шаров, наудачу выбирают один шар. Чему равна вероятность того, что этот шар "--- белый?
\end{problem}

%% Решение задания №1
\begin{proof}
	Пусть $ A $ "--- событие, что достали белый шар.
	Количество всех исходов будет равно: $ \#\Omega \hm= K + L $.
	
	Тогда количество благоприятных исходов (наступления события $ A $) равно: $ \#A = K $.
	
	Отсюда получаем, что вероятность наступления события $ A $ равна:
	\[ \Prob A = \cfrac{\#A}{\#\Omega} = \cfrac{K}{K + L}. \]
\end{proof}

%%%%%%%%%%%%%% ЗАДАНИЕ №2 %%%%%%%%%%%%%%
%% Условие задания №2
\begin{problem}
	Распределение случайной величины $ \xi $ задано таблицей:
	\begin{center}
		\begin{tabular}{|c|c|c|c|c|c|}
		\hline
		$ \xi $ & 1 & 2 & 4 & 6 & $ \Sigma $ \\
		\hline
		$ \mathbb{P} $ & 0,1 & 0,2 & 0,6 & 0,1 & 1 \\
		\hline
		\end{tabular}
	\end{center}
	Вычислить $ \Expect\xi $, $ \Variance\xi $, $ \Entropy\xi $ (в натах) и распределение $ \eta = \sin(\pi\xi/3) $.
\end{problem}

%% Решение задания №2
\begin{proof}
	Математическое ожидание дискретной случайной величины $ \xi $ задаётся формулой:
	\[ \Expect\xi = \sum_{i \colon p_i > 0}a_ip_i. \]
	Отсюда получаем:
	\[ \Expect\xi = 1 \cdot 0,1 + 2 \cdot 0,2 + 4 \cdot 0,6 + 6 \cdot 0,1 = 3,5. \]
	Дисперсия дискретной случайной величины $ \xi $ задаётся формулой:
	\[ \Variance\xi = \sum_{i \colon p_i > 0}(a_i - \mathbb E\xi)^2p_i. \]
	Отсюда получаем:
	\[ \Variance\xi = (1 - 3,5)^2 \cdot 0,1 + (2 - 3,5)^2 \cdot 0,2 + (4 - 3,5)^2 \cdot 0,6 + (6 - 3,5)^2 \cdot 0,1 = 1,85. \]
	Энтропия дискретной случайной величины $ \xi $ задаётся формулой:
	\[ \Entropy\xi = -\sum_{i \colon p_i > 0}p_i\log_bp_i. \]
	Необходимо вычислить энтропию в натах, т~е. $ b = e $. Получим:
	\[ \Entropy\xi = -(0,1 \cdot \ln0,1 + 0,2 \cdot \ln0,2 + 0,6 \cdot \ln0,6 + 0,1 \cdot \ln0,1) \approx 1,0889. \]
	Носитель случайной величины $ \xi $ имеет вид: $ \supp \xi = \lbrace 1, 2, 4, 6 \rbrace $. Тогда носитель случайной величины $ \eta $ будет иметь вид: $ \supp \eta = \left\lbrace -\frac{\sqrt{3}}{2}, 0, \frac{\sqrt{3}}{2} \right\rbrace  $. Найдём вероятности появления каждого числа:
	\begin{align*}
		\Prob(\eta = -\sqrt{3}/2) &= \Prob(\xi = 4) = 0,6. \\
		\Prob(\eta = 0) &= \Prob(\xi = 6) = 0,1. \\
		\Prob(\eta = \sqrt{3}/2) &= \Prob(\xi = 1) + \Prob(\xi = 2) = 0,3.
	\end{align*}
	Таким образом, можно записать распределение случайной величины $ \eta $ в виде таблицы:
	\begin{center}
		\begin{tabular}{|c|c|c|c|c|}
			\hline
			$ \eta $  & $ -\sqrt{3}/2 $ & 0   & $ \sqrt{3}/2 $ & $ \Sigma $ \\ \hline
			$ \Prob $ & 0,6             & 0,1 & 0,3            & 1          \\ \hline
		\end{tabular}
	\end{center}
\end{proof}

%%%%%%%%%%%%%% ЗАДАНИЕ №3 %%%%%%%%%%%%%%
%% Условие задания №3
\begin{problem}
	Условие задачи №3.
\end{problem}

%% Решение задания №3
\begin{proof}
	Решение задачи №3.
\end{proof}